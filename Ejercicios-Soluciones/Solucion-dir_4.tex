\documentclass[twoside,10.5pt]{article}%
\usepackage{minted}   % importamos el paquete minted                        
\usepackage{mathrsfs}% 
\usepackage[sc]{mathpazo}                                          
\usepackage{pifont}%                                             
\usepackage{amsmath}%                                            
\usepackage{amsthm}%                                             
\usepackage{txfonts}%                                            
\usepackage{geometry}%                                           
\usepackage{latexsym}%                                           
\usepackage{amssymb}%                                            
\usepackage{graphicx}%                                           
\usepackage{geometry}%                                           
\usepackage{xcolor} %                                            
\geometry{paperheight=28.5cm,paperwidth=21cm,top=2.5cm,%         
bottom=2.6cm,left=2.5cm,right=2.5cm,headheight=0.8cm,%             
headsep=0.9cm,textheight=20cm,footskip=1cm}%                   
\setlength{\parindent}{0pt} \setlength{\parskip}{5pt}%           
\renewcommand{\baselinestretch}{1.0}%                            *                                                        *
\pagestyle{empty}
\begin{document}
\begin{center}
{\LARGE{Soluciones de la dirigida 4}}\\[20pt]
\end{center}

\vspace{0.3cm}




\begin{minted}{c}
/* Este programa pide ingresar un entero positivo "n" mayor que 2 
 * y muestra todos los primos menores o iguales a "n" */
 
#include <stdio.h>

void rombo (int n);
void tri1 (int n);
void tri2 (int n);

void main (void)
{
	int n;
	
	printf("\n Ingrese un entero mayor que 1: ");
    scanf("%d",&n);
    
    printf("\n Rombo de diagonal %d:\n", 2*n + 1);
	rombo(n);
	
	printf("\n Triangulo de altura %d y ancho %d:\n", 2*n - 1, n);
	tri1(n);
	
	printf("\n Triangulo de altura %d y ancho %d:\n", n, 2*n - 1);
	tri2(n);

	printf("\n");
}


void rombo (int n)
{
	int x, y;
    for (y = -n; y <= n ; y++)
    {
        for (x = n; x >= -n; x--)
            if ( (abs(x) + abs(y)) <= n ) printf("*"); 
            else printf(" ");
        printf("\n");
    }
}

void tri1 (int n)
{
	int x, y;
    for (y = 1; y < n ; y++)
    {
        for (x = 1; x <= y; x++)
            printf("*");
        printf("\n");
    }
    for (  ; y > 0 ; y--)
    {
        for (x = 1; x <= y; x++)
            printf("*");
        printf("\n");
    }
}

void tri2 (int n)
{
	int x, y;
    for (y = n-1; y >= 0; y--)
    {
        for (x = -n+1; x <= n-1; x++)
            if ( (abs(x) + abs(y)) <= n-1 ) printf("*"); 
            else printf(" ");
        printf("\n");
    }
}
\end{minted}

\vspace{0.3cm}


\begin{minted}{c}
/* Este programa pide ingresar dos enteros positivos y muestra
 el MCD y el MCM de ambos, empleando funciones */
 
#include <stdio.h>

// prototipos de funciones
int mcd (int a, int b); 
int mcm (int a, int b);

int main (void)
{
	int a,b;
	
	printf("\n a: ");
	scanf("%d", &a);
	printf(" b: ");
	scanf("%d", &b);

	printf(" mcd = %d.\n",mcd(a,b));
	printf(" mcm = %d.\n",mcm(a,b));
	printf("\n");
	return 7;		
}

// funcion que retorna el MCD si se ingresan dos enteros positivos
int mcd (int a, int b)
{
	int d = 1, m;
	while (d <= a && d <= b)
	{
		if (a % d == 0  &&  b % d == 0) m = d;
		d++;
	}
	return m;
}

// funcion que retorna el MCM si se ingresan dos enteros positivos
int mcm (int a, int b)
{
	int m = mcd(a,b);
	return (a*b)/m;
}
\end{minted}

\vspace{0.3cm}


\begin{minted}{c}
#include <stdio.h>
#include <stdlib.h>
 
int dot_product(int *, int *, size_t);
 
int
main(void)
{
        int a[3] = {1, 3, -5};
        int b[3] = {4, -2, -1};
 
        printf("%d\n", dot_product(a, b, sizeof(a) / sizeof(a[0])));
 
        return EXIT_SUCCESS;
}
 
int
dot_product(int *a, int *b, size_t n)
{
        int sum = 0;
        size_t i;
 
        for (i = 0; i < n; i++) {
                sum += a[i] * b[i];
        }
 
        return sum;
}
\end{minted}

\vspace{0.3cm}



\begin{minted}{c}
/* Este programa pide ingresar un entero positivo "n" mayor que 2 
 * y muestra todos los primos menores o iguales a "n" */
 
#include <stdio.h>

int esprimo (int N);

void main (void)
{
	int n, i, d, cont;
	
	printf("\n Ingrese un entero n mayor que 2: ");
	scanf("%d", &n);
	printf(" Los primos menores o iguales a %d son: 2", n);
	
	for(i = 3; i < n; i++)
		if (esprimo(i) == 1) printf(", %d",i);
	printf(".\n\n");
}


int esprimo (int N)
{
	int d = 2, cont = 0;
	while (d < N && cont == 0)
	{
		if (N % d == 0) cont++;
		d++;
	}
	if (cont == 0) return 1;
	else return 0;
}
\end{minted}


\vspace{0.3cm}

\begin{minted}{c}
#include<stdio.h>
 
typedef struct{
	float i,j,k;
	}Vector;
 
Vector a = {3, 4, 5},b = {4, 3, 5},c = {-5, -12, -13};
 
float dotProduct(Vector a, Vector b)
{
	return a.i*b.i+a.j*b.j+a.k*b.k;
}
 
Vector crossProduct(Vector a,Vector b)
{
	Vector c = {a.j*b.k - a.k*b.j, a.k*b.i - a.i*b.k, a.i*b.j - a.j*b.i};
 
	return c;
}
 
float scalarTripleProduct(Vector a,Vector b,Vector c)
{
	return dotProduct(a,crossProduct(b,c));
}
 
Vector vectorTripleProduct(Vector a,Vector b,Vector c)
{
	return crossProduct(a,crossProduct(b,c));
}
 
void printVector(Vector a)
{
	printf("( %f, %f, %f)",a.i,a.j,a.k);
}
 
int main()
{
	printf("\n a = "); printVector(a);
	printf("\n b = "); printVector(b);
	printf("\n c = "); printVector(c);
	printf("\n a . b = %f",dotProduct(a,b));
	printf("\n a x b = "); printVector(crossProduct(a,b));
	printf("\n a . (b x c) = %f",scalarTripleProduct(a,b,c));
	printf("\n a x (b x c) = "); printVector(vectorTripleProduct(a,b,c));
 
	return 0;
}
\end{minted}
\end{document}