\documentclass[twoside,10.5pt]{article}%
\usepackage{minted}   % importamos el paqeuete minted                      
\usepackage{mathrsfs}% 
\usepackage[sc]{mathpazo}                                          
\usepackage{pifont}%                                             
\usepackage{amsmath}%                                            
\usepackage{amsthm}%                                             
\usepackage{txfonts}%                                            
\usepackage{geometry}%                                           
\usepackage{latexsym}%                                           
\usepackage{amssymb}%                                            
\usepackage{graphicx}%                                           
\usepackage{geometry}%                                           
\usepackage{xcolor} %                                            
\geometry{paperheight=28.5cm,paperwidth=21cm,top=2.5cm,%         
bottom=2.6cm,left=2.5cm,right=2.5cm,headheight=0.8cm,%             
headsep=0.9cm,textheight=20cm,footskip=1cm}%                   
\setlength{\parindent}{0pt} \setlength{\parskip}{5pt}%           
\renewcommand{\baselinestretch}{1.0}%                            *                                                        *
\pagestyle{empty}
\begin{document}
\begin{center}
{\LARGE{Algunas soluciones de la  dirigida 6}}\\[20pt]
\end{center}

\vspace{0.3cm}

\begin{enumerate}
\item Desarrollemos un programa que muestre el tri\'angulo de Pascal


\begin{minted}{c}
/*
Un programa para generar el triangulo de Pascal
usando un arreglo  unidimensional
*/


#include<stdio.h>
void main()

{
        int arr[30],temp[30],i,j,k,l,numero;           //usando dos arreglos 
      
       printf("Ingresa el numero de lineas a ser mostrada: ");
       scanf("%d",&numero);

       temp[0]=1;
       arr[0]=1;      //  Inicializamos los arreglos

       for(j=0;j<numero;j++)
      printf(" ");

       printf(" 1\n");

       for(i=1;i<numero;i++)
       {
               for(j=0;j<numero-i;j++)
        printf(" ");                        //  Espacios y unos 

        for(k=1;k<i;k++)
        {
                   arr[k]=temp[k-1]+temp[k];  //  Usamos la propiedad del triangulo de Pascal
         }                                    // cada numero es la suma de los dos numeros 
         							//que tiene encima
                arr[i]=1;

                for(l=0;l<=i;l++)
        {
                  printf("%2d",arr[l]);
                  temp[l]=arr[l];
         }						// Mostramos los valores

                printf("\n");
        }
}
\end{minted}
 \item Escribe un algoritmo que genere una permutaci\'on aleatoria de los elementos del arreglo.
 
 \begin{minted}{c}
 /*
  Un programa que muestra una permutacion de un arreglo dado.
 */
#include <stdio.h>
#include <time.h>

void muestraArreglo(char *arreglo[], int s);
void intercambio(int i, int j, char *arreglo[]);
void per(char *arreglo[], int s);


int main()
{
    char *palabras[] = {"a","b","c","d","e",
                     "f","g","h","i","j", "k", "l", "m","n",
		    "o", "p", "q", "r", "s","t","u", "v", "w",
		    "x", "y", "z"};
    int s = sizeof(palabras)/sizeof(palabras[0]);
    per(palabras,s);
    muestraArreglo(palabras,s);
    return 0;
}

//Recorremos el arreglo y  en cada interacion generamos un numero aleatorio 
//entre  0 y i y  intercambiamos este con el numero  en la  posicion i.
void per(char *arreglo[], int s) {
    int i,r;
    srand(time(NULL)); 
    for (i = s-1; i >= 0; i--) {
        r = rand() % (i + 1);
        intercambio(i,r,arreglo);
    }
}

void intercambio(int i, int j, char *arreglo[]) {
    char *temp = arreglo[i];
    arreglo[i] = arreglo[j];
    arreglo[j] = temp;
}

void muestraArreglo(char *arreglo[], int s) {
    int i;
    for (i=0; i<s; i++) {
        printf("%s \n", arreglo[i]);
    }
}
 \end{minted}



\item  Escribe un programa que, dados los contenidos en un vector, calcule e imprima el m\'aximo, el m\'inimo, la media, y la desviaci\'on est\'andar.

\begin{minted}{c}
#include <stdio.h>
#include <math.h>

#define MAX 50     /* Maximo elemento de un vector */

int main()
{
  int i, n;
  float arreglo[MAX];
  float  max, min, sum, cuadrados, media, dv;
  do
  {
    printf("Cuantos datos deseas ingresar? (2..%i): ", MAX);
    scanf("%i", &n);
  }
  while ((n < 2) || (n > MAX));
  for (i = 0; i < n; i++)
  {
    printf("Introduzca el numero %i:", i +1);
    scanf("%f", &arreglo[i]); 
  }
  
  /* Calculamos algunos estadisticos */
  
  max = arreglo[0];
  min = arreglo[0];
  sum = arreglo[0];
  
  cuadrados = arreglo[0]*arreglo[0];
  for (i = 1; i < n; i++)
  {
    sum += arreglo[i];
    if (arreglo[i]> max)
      max = arreglo[i];
    else if (arreglo [i] < min)
      min = arreglo[i];
    cuadrados += arreglo[i] * arreglo[i];
  }
  
  media= sum/n;
  dv  =sqrt(cuadrados/n - media*media);
  printf("Maximo = %6.2f\n", max);
  printf("Minimo = %6.2f\n", min);
  printf("Media = %6.2f\n", media);
  printf("Desv    = %6.2f\n", dv);
  
  return 0;
  
}
\end{minted}

\item \textbf{Algoritmos de Ordenamiento}

1) Escribe un programa que ordene $n$ n\'umeros de forma ascendente usando el ordenamiento de burbuja.

\begin{minted}{c}
/*
 * Un programa para ordenar n numeros, usando el ordenamiento
 * burbuja. Imprimimos el vector inicial y luego ordenado.
 */
#include <stdio.h>
#define MAX 10
 
void main()
{
    int arreglo[MAX];
    int i, j, numero, temp;
 
    printf("Ingresamos la longitud del vector \n");
    scanf("%d", &numero);
    printf("Ingresamos los elementos uno a uno \n");
    for (i = 0; i < numero; i++)
    {
        scanf("%d", &arreglo[i]);
    }
    printf("El vector de entrada es   \n");
    for (i = 0; i < numero; i++)
    {
        printf("%d\n", arreglo[i]);
    }
    /*   Empezamo el ordenamiento burbuja */
    for (i = 0; i < numero; i++)
    {
        for (j = 0; j < (numero - i - 1); j++)
        {
            if (arreglo[j] > arreglo[j + 1])
            {
                temp = arreglo[j];
                arreglo[j] = arreglo[j + 1];
                arreglo[j + 1] = temp;
            }
        }
    }
    printf("El vector ordenado es...\n");
    for (i = 0; i < numero; i++)
    {
        printf("%d\n", arreglo[i]);
    }
}
\end{minted}

2) Escribe un programa que ordene $n$ n\'umeros de forma ascendente usando el ordenamiento por inserci\'on.


\begin{minted}{c}
/*
 * Un programa en C, que ordena elementos numeroericos por insercion 
 * 
 */

#include<stdio.h>

int main() {
   int i, j, numero, temp, vector[20];

   printf("Ingresa la longitud del vector:\n ");
   scanf("%d", &numero);

   printf("Ingresa  %d elementos: \n", numero);
   for (i = 0; i < numero; i++) {
      scanf("%d", &vector[i]);
   }

   for (i = 1; i < numero; i++) {
      temp = vector[i];
      j = i - 1;
      while ((temp < vector[j]) && (j >= 0)) {
         vector[j + 1] = vector[j];
         j = j - 1;
      }
      vector[j + 1] = temp;
   }

   printf("Despues del ordenamiento:\n ");
   for (i = 0; i < numero; i++) {
      printf("%d \n", vector[i]);
   }

   return 0;
}
\end{minted} 

\item \mbox{Implementa} el \mbox{algoritmo} en C y encuentra un elemento de un arreglo ordenado, usando una \textit{b\'usqueda \mbox{binaria}.}


\begin{minted}{c}
/*
 * Este es un programa en C  que acepta n numeroeros almacenados en  
 * orden ascendente  y busca un numeroero usando la bisqueda binaria.
 * Muestra un mensaje de exito o de fallo.
 */
#include <stdio.h>
 
void main()
{
    int vector[10];
    int i, j, numero, temp, keynumero;
    int menor, medio, mayor;
 
    printf("Ingresa la longitud del vector \n");
    scanf("%d", &numero);
    printf("Ingresa los elementos uno a uno  \n");
    for (i = 0; i < numero; i++)
    {
        scanf("%d", &vector[i]);
    }
    printf("Elementos del vector de entrada son \n");
    for (i = 0; i < numero; i++)
    {
        printf("%d\n", vector[i]);
    }
    /*  El ordenamiento de burbuja empieza */
    for (i = 0; i < numero; i++)
    {
        for (j = 0; j < (numero - i - 1); j++)
        {
            if (vector[j] > vector[j + 1])
            {
                temp = vector[j];
                vector[j] = vector[j + 1];
                vector[j + 1] = temp;
            }
        }
    }
    printf("El vector ordenado es ...\n");
    for (i = 0; i < numero; i++)
    {
        printf("%d\n", vector[i]);
    }
    printf("Ingresa el elemento a ser buscado \n");
    scanf("%d", &keynumero);
    /*  La busqueda binaria empieza*/
    menor = 1;
    mayor = numero;
    do
    {
        medio = (menor + mayor) / 2;
        if (keynumero < vector[medio])
            mayor = medio - 1;
        else if (keynumero > vector[medio])
            menor = medio + 1;
    } while (keynumero != vector[medio] && menor <= mayor);
    if (keynumero == vector[medio])
    {
        printf("Busqueda exitosa \n");
    }
    else
    {
        printf("Ooops fallo la busqueda \n");
    }
}
\end{minted}


\item Desarrolla un algoritmo para representar un sistema lineal de ecuaciones en forma matricial.

\begin{minted}{c}
/*
 *  Programa en C que expresa un sistema lineal
 *  en su forma matricial 
 */

#include <stdio.h>
#include <stdlib.h>
int main(void) {
    char var[] = { 'x', 'y', 'z', 'w' };
    printf(" Ingresamos el numero de variables en las ecuaciones: ");
    int n;
    scanf("%d", &n);
    printf("\n Ingresamos los coeficientes de cada variable para cada ecuacion");
    printf("\nax + by + cz + ... = d\n");
    int mat[n][n];
    int constantes[n][1];    
    int i,j;
    for (i = 0; i < n; i++) {
        for (j = 0; j < n; j++) {
            scanf("%i", &mat[i][j]);
        }
        scanf("%i", &constantes[i][0]);
    }
    
    printf("La matriz representacion es:\n");
    for (i = 0; i < n; i++) {
        for (j = 0; j < n; j++) {
            printf(" %i", mat[i][j]);
        }
        printf(" %c", var[i]);
        printf(" = %i", constantes[i][0]);
        printf("\n");
    }
    return 0;
}
\end{minted}

\item Escribe un programa para calcular la suma o sustracci\'on y traza de dos matrices.

\begin{minted}{c}
/*
 * Programa en C que lee dos matrices A y B  y lleva a cabo la adicion 
 * o resta de  A y  B. Tambien encuentra la traza de la matriz resultante.
 * Muestra las  matrices, sus sumas o resta y la traza
 */
#include <stdio.h>
void traza(int arr[][10], int m, int n);
 
void main()
{
    int matriz1[10][10], matriz2[10][10], matrizsuma[10][10],
    matrizresta[10][10];
    int i, j, m, n, opciones;
 
    printf("Ingresa el orden de las matrices  matriz1 y  matriz2 \n");
    scanf("%d %d", &m, &n);
    printf("Ingresa los elementos de matriz1 \n");
    for (i = 0; i < m; i++)
    {
        for (j = 0; j < n; j++)
        {
            scanf("%d", &matriz1[i][j]);
        }
    }
    printf("Los elementos de la primera matriz  son \n");
    for (i = 0; i < m; i++)
    {
        for (j = 0; j < n; j++)
        {
            printf("%3d", matriz1[i][j]);
        }
        printf("\n");
    }
    printf("Ingresa los elementos de  matriz2 \n");
    for (i = 0; i < m; i++)
    {
        for (j = 0; j < n; j++)
        {
            scanf("%d", &matriz2[i][j]);
        }
    }
    printf("Los elementos de la segunda matriz  son \n");
    for (i = 0; i < m; i++)
    {
        for (j = 0; j < n; j++)
        {
            printf("%3d", matriz2[i][j]);
        }
        printf("\n");
    }
    printf("Ingresa tus opciones: 1 para la suma  y  2 para la resta \n");
    scanf("%d", &opciones);
    switch (opciones)
    {
    case 1:
        for (i = 0; i < m; i++)
        {
            for (j = 0; j < n; j++)
            {
                matrizsuma[i][j] = matriz1[i][j] + matriz2[i][j];
            }
        }
        printf("La matriz suma es \n");
        for (i = 0; i < m; i++)
        {
            for (j = 0; j < n; j++)
            {
                printf("%3d", matrizsuma[i][j]) ;
            }
            printf("\n");
        }
        traza (matrizsuma, m, n);
        break;
    case 2:
        for (i = 0; i < m; i++)
        {
            for (j = 0; j < n; j++)
            {
                matrizresta[i][j] = matriz1[i][j] - matriz2[i][j];
            }
        }
        printf("La matriz resta es \n");
        for (i = 0; i < m; i++)
        {
            for (j = 0; j < n; j++)
            {
                printf("%3d", matrizresta[i][j]) ;
            }
            printf("\n");
        }
        traza (matrizresta, m, n);
        break;
    }
 
}
/* Funcion para encontrar la traza de una funcion dada y que se imprime  */
void traza (int arr[][10], int m, int n)
{
    int i, j, traza = 0;
    for (i = 0; i < m; i++)
    {
        for (j = 0; j < n; j++)
        {
            if (i == j)
            {
                traza = traza + arr[i][j];
            }
        }
    }
    printf("La traza de la matriz  resultantante es = %d\n", traza);
}
\end{minted}



\item Escribe un programa para multiplicar dos matrices, usando recursi\'on. 


\begin{minted}{c}
/*
 * Un programa en C, para multiplicar matrices, usando  Recursion
 */
#include <stdio.h>
 
void multiplicacion(int, int, int [][10], int, int, int [][10], int [][10]);
void muestra_multiplicacion(int, int, int[][10]);
 
int main()
{
    int a[10][10], b[10][10], c[10][10] = {0};
    int m1, n1, m2, n2, i, j, k;
 
    printf("Ingresa las filas y columnas de la matriz A : ");
    scanf("%d%d", &m1, &n1);
    printf("Ingresa las filas y columnas de la matriz B: ");
    scanf("%d%d", &m2, &n2);
    if (n1 != m2)
    {
        printf("La multiplicacion de estas matrices no es posible .\n");
    }
    else
    {
        printf("Ingresa los elementos de la matriz A :\n");
        for (i = 0; i < m1; i++)
        for (j = 0; j < n1; j++)
        {
            scanf("%d", &a[i][j]);
        }
        printf("\n Ingresa los elementos en la matriz B:\n");
        for (i = 0; i < m2; i++)
        for (j = 0; j < n2; j++)
        {
            scanf("%d", &b[i][j]);
        }
        multiplicacion(m1, n1, a, m2, n2, b, c);
    }
    printf("La multiplicacion de A y  B es:\n");
    muestra_multiplicacion(m1, n2, c);
}
 
void multiplicacion (int m1, int n1, int a[10][10], int m2, int n2, int b[10][10], 
                        int c[10][10])
{
    static int i = 0, j = 0, k = 0;
 
    if (i >= m1)
    {
        return;
    }
    else if (i < m1)
    {
        if (j < n2)
        {
            if (k < n1)
            {
                c[i][j] += a[i][k] * b[k][j];
                k++;
                multiplicacion(m1, n1, a, m2, n2, b, c);
            }
            k = 0;
            j++;
            multiplicacion(m1, n1, a, m2, n2, b, c);
        }
        j = 0;
        i++;
        multiplicacion(m1, n1, a, m2, n2, b, c);
    }
}
 
void muestra_multiplicacion(int m1, int n2, int c[10][10])
{
    int i, j;
 
    for (i = 0; i < m1; i++)
    {
        for (j = 0; j < n2; j++)
        {
            printf("%d  ", c[i][j]);
        }
        printf("\n");
    }
}
\end{minted}






\item Escribamos un posible algoritmo en C, de la eliminaci\'on de Gauss-Jordan (Posible soluci\'on)

\begin{minted}{c}
/*
 * Un programa en C para calcular el metodo de 
 * Gauss-Jordan
*/

#include<stdio.h>
int main()
{
    int i,j,k,n;
    float A[20][20],c,x[10];
    printf("\nIngresa la dimension de la matriz : ");
    scanf("%d",&n);
    printf("\nIngresa los elementos de la matriz aumentada :\n");
    for(i=1; i<=n; i++)
    {
        for(j=1; j<=(n+1); j++)
        {
            printf(" A[%d][%d]:", i,j);
            scanf("%f",&A[i][j]);
        }
    }
    /* Ahora encontramos los elementos de la matriz diagonal  */
    for(j=1; j<=n; j++)
    {
        for(i=1; i<=n; i++)
        {
            if(i!=j)
            {
                c=A[i][j]/A[j][j];      // Como calcular el pivote
                for(k=1; k<=n+1; k++)
                {
                    A[i][k]=A[i][k]-c*A[j][k];
                }
            }
        }
    }
    printf("\nLa solucion es:\n");
    for(i=1; i<=n; i++)
    {
        x[i]=A[i][n+1]/A[i][i];
        printf("\n x%d=%f\n",i,x[i]);
    }
    return 0;
}
\end{minted}



\item Crea un algoritmo que cargue dos cadenas de caracteres y las concatene. \textbf{Nota:} \textit{Usar la funci\'on \texttt{strcat()}}.


\begin{minted}{c}
/* Un programa en C para concatenar cadenas
 * de caracteres
 */

#include <stdio.h>
#include <string.h>

#define MAX_LEN 100

typedef char cadena_1[MAX_LEN + 1];

int main()
{
  cadena_1 c1 = " Hola ";
  cadena_1 c2 = " Somos  Python y C";
  
  printf("c1: %s\n", c1);
  printf("c2: %s\n", c2);
  
  strcat(c1, c2);
  printf("Concatenacion de dos cadenas: %s\n", c1);  // uso de strcat()
  
  return 0;
  
}
\end{minted}

\item Desarrolla un programa, que ingrese una cadena de caracteres, y devuelva las versiones en min\'uscula y may\'uscula de esa cadena. \textbf{Nota:} \textit{Usar la funci\'on \texttt{ fgets(), tolower(), toupper()}}.
\begin{minted}{c}
#include <stdio.h>
#include <string.h>
#include <ctype.h>
int main()
{
    char str[255];
    printf("Ingresa una cadena : "); fgets(str,255, stdin);
    printf("Esta es tu cadena que has escrito: %s\n", str);
    //Ahora recorramos esta cadena y que se convierta toda en minuscula
    int i;
    for(i = 0; str[i]; i++)
    {
        str[i] = tolower(str[i]);     // Usamos la funcion tolower()
    }
    printf("Aqui tienes la version en minuscula de tu cadena : %s\n", str);    
    //Ahora recorramos esta cadena y que se convierta toda en minuscula
    int j;
    for(j = 0; str[j]; j++)
    {
        str[j] = toupper(str[j]); //Usamos la funcion toupper()
    }
    printf("Aqui tienes la version en mayuscula de tu cadena: %s\n", str);    
    return 0;
}
\end{minted}

\end{enumerate}
\end{document}