\documentclass[twoside,10.5pt]{article}%
\usepackage{minted}   % importamos el paquete minted                       
\usepackage{mathrsfs}% 
\usepackage[sc]{mathpazo}                                          
\usepackage{pifont}%                                             
\usepackage{amsmath}%                                            
\usepackage{amsthm}%                                             
\usepackage{txfonts}%                                            
\usepackage{geometry}%                                           
\usepackage{latexsym}%                                           
\usepackage{amssymb}%                                            
\usepackage{graphicx}%                                           
\usepackage{geometry}%                                           
\usepackage{xcolor} %                                            
\geometry{paperheight=28.5cm,paperwidth=21cm,top=2.5cm,%         
bottom=2.6cm,left=2.5cm,right=2.5cm,headheight=0.8cm,%             
headsep=0.9cm,textheight=20cm,footskip=1cm}%                   
\setlength{\parindent}{0pt} \setlength{\parskip}{5pt}%           
\renewcommand{\baselinestretch}{1.0}%                            *                                                        *
\pagestyle{empty}
\begin{document}
\begin{center}
{\LARGE{Solucion de la dirigida 3}}\\[20pt]
\end{center}

\vspace{0.3cm}




\begin{minted}{c}
/* Lee los numeros desde el teclado
 * y imprime su promedio
 */

#include <stdio.h>
int main( )
{
  double x = 0.0, sum = 0.0;
  int count = 0;
  
  printf( "\t--- Calculando Promedios ---\n" );
  printf( "\nIngresar algunos numeros:\n"
"(Escribe una letra al final de tu entrada )\n" );
  
while (scanf( "%lf", &x ) == 1 )
{
  sum += x;
  ++count;
}

if ( count == 0 )
  printf( "No entrada de datos!\n" );
else
printf( "El promedio de numeros es %.2f\n", sum/count );
return 0;
}
\end{minted}

\vspace{0.3cm}

\begin{minted}{c}
#include <stdio.h>
main ()
{
  int i, s, salida_numero;
  int revnum = 0;
  printf("Ingresar un numero \n");
  scanf("%i", &i);
  while(i > 0)
  {
    salida_numero = i% 10;  //quitamos el ultimo digito
    revnum = revnum * 10 + salida_numero;
    //printf("%i\n", salida_numero);  
    i = i/10;
    
    s = revnum;
   // printf("\n\n");
    while(revnum > 0)
    {
      salida_numero = revnum %10; //quitamos el ultimo digito
      //printf("%i\n", salida_numero);
      revnum /= 10;
    }
    
   // printf("\n\n");
    
    /*Ahora imprimimos cada digito con palabras en ingles
     */
    
    while (s >0)
    {
      salida_numero = s % 10;   //quitamos el ultimo digito
      printf("%i\t", salida_numero);
      //printf("%s\t", palabras[salida_numero]);
      
      switch(salida_numero)
      {
	case 0:
	  printf("Zero\n");
	  break;
	case 1:
	  printf("One\n");
	  break;
	case 2:
	  printf("Two\n");
	  break;
	case 3:
	  printf("Three\n");
	  break;
	case 4:
	  printf("Four\n");
	  break;
	case 5:
	  printf("Five\n");
	  break;
	case 6:
	  printf("Six\n");
	  break;
	case 7:
	  printf("Seven\n");
	  break;
	case 8:
	  printf("Eight\n");
	  break;
	case 9:
	  printf("Nine\n");
	  break;
      }
      
      s /= 10;   //Dividimos el actual numero por 10
    }
    
  }
}
\end{minted}

\vspace{0.3cm}

\begin{minted}{c}
/*
 * Calculamos he imprimimos todos los numeros primos
 */

#include <stdio.h>

int main()
{
  int numero;
  int divisor;
  
  /*
   * Uno y dos son faciles
   */
  
  printf("1\n2\n");
  
  
  /* Solo el numero par 2 es primo...aprovechemos eso
   */
  
  for(numero = 3; numero <= 30; numero = numero + 2){
    /*
     * Vemos si el algun divisor desde 3 hasta el numero divide al numero
     */
    for(divisor = 3; divisor < numero; divisor = divisor + 2){
      if (numero %divisor ==0)
	break;
    }
    
    /*Si el ciclo arriba para debido a que el divisor es 
     * muy grande, tenemos entonces un numero primo
     */
    
    
    if(divisor >=numero)
      printf("%d\n", numero);
  }
  
}
\end{minted}



\vspace{0.3cm}


\begin{minted}{c}
/* Clasificando los tipos de triangulo dado
 * la longitud de sus lados
 */

#include <stdlib.h>
#include <stdio.h>

int main()
{
  float  a;
  float  b;
  float  c;
  float  temp;
  
  /*
   Solicitamos y leemos los datos
   */
  
  printf("Ingresar las longitudes de los tres lados del triangulo: ");
  scanf("%f %f %f", &a, &b, &c);
  
  /*
   * Reordenamos los valores tal que a es el de mayor longitud
   * y c es de menor longitud
   */
  
  
if( a < b ){
  
    temp = a;
    a = b;
    b = temp;
}
if( a < c ){
   temp = a;
   a = c;
   c = temp;
}
if( b < c ){
   temp = b;
   b = c;
   c = temp;
}

/* Ahora veamos que tipo de triangulo es. Notamos que si
 * uno de los lados es <= 0 o si dos de lados juntos es 
 * menor que el mayor, entonces no son triangulos
 */

if (c <= 0 || 	 b +c  < a)
  printf("No es un triangulo.\n");

else if(a == b && b==c)
  printf("Equilatero.\n");
else if	(a ==b || b ==c)
  printf("Isosceles.\n");
else
  printf("Escaleno.\n");
return EXIT_SUCCESS;
}
\end{minted}

\vspace{0.3cm}


\begin{minted}{c}
/* Programa para diseñar un diamante
 * con asteriscos 
 */

#include <stdio.h>
 
int main()
{
  int n, c, k, espacio = 1;
 
  printf("Ingresamos el numero de filas \n");
  scanf("%d", &n);
 
  espacio = n - 1;
 
  for (k = 1; k <= n; k++)
  {
    for (c = 1; c <= espacio; c++)
      printf(" ");
 
    espacio--;
 
    for (c = 1; c <= 2*k-1; c++)
      printf("*");
 
    printf("\n");
  }
 
  espacio = 1;
 
  for (k = 1; k <= n - 1; k++)
  {
    for (c = 1; c <= espacio; c++)
      printf(" ");
 
    espacio++;
 
    for (c = 1 ; c <= 2*(n-k)-1; c++)
      printf("*");
 
    printf("\n");
  }
 
  return 0;
}
\end{minted}

\vspace{0.3cm}

\begin{minted}{c}
/*
 *  Un programa donde se ingresan datos y encontramos la media,
 * la  varianza
 * y la desviacion estandar 
 */


#include <stdio.h>
#include <math.h>
#define MAXSIZE 10
 
void main()
{
    float x[MAXSIZE];
    int  i, n;
    float promedio, varianza, std, sum = 0, sum1 = 0;
 
    printf("Ingresar el valor de N \n");
    scanf("%d", &n);
    printf("Ingresar  %d numeros reales \n", n);
    for (i = 0; i < n; i++)
    {
        scanf("%f", &x[i]);
    }
    /*  Calculamos la suma de todos los elementos */
    for (i = 0; i < n; i++)
    {
        sum = sum + x[i];
    }
    promedio = sum / (float)n;
    /*  Calculamos la  varianza  y la desviacion estandar */
    for (i = 0; i < n; i++)
    {
        sum1 = sum1 + pow((x[i] - promedio), 2);
    }
    varianza = sum1 / (float)n;
    std = sqrt(varianza);
    printf("El promedio de todos los elementos = %.2f\n", promedio);
    printf("La varianza de todos los elementos = %.2f\n", varianza);
    printf("La desviacion estandar es = %.2f\n", std);
}

\end{minted}


\vspace{0.5cm}

\begin{minted}{c}

/*
 * Un programa en C, para calcular la suma de 1^2 + 2^2 + …. + n^2.
 */

#include <stdio.h>
 
int main()
{
    int numero, i;
    int suma = 0;
 
    printf("Ingreso el maximo valor  para la suma: ");
    scanf("%d", &numero);
    suma = (numero * (numero + 1) * (2 * numero + 1 )) / 6;
    printf("La suma de los elementos de arriba es : ");
    for (i = 1; i <= numero; i++)
    {
        if (i != numero)
            printf("%d^2 + ", i);
        else
            printf("%d^2 = %d ", i, suma);
    }
    return 0;
}

\end{minted}
\vspace{0.5cm}

\begin{minted}{c}

/*
 * Un programa para calcular el valor del coseno en terminos de series
 * dado un grado de precision
 * 
 */
#include <stdio.h>
#include <math.h>
#include <stdlib.h>
 
void main()
{
    int n, x1;
    float presc, term, denominador, x, cosx, cosval;
 
    printf("Ingrese el valor de x (en grados) \n");
    scanf("%f", &x);
    x1 = x;
    /*  Convirtiendo grados a radianes */
    x = x * (3.142 / 180.0);
    cosval = cos(x);
    printf("Ingresar el grado de precision para la funcion \n");
    scanf("%f", &presc);
    term = 1;
    cosx = term;
    n = 1;
    do
    {
        denominador = 2 * n * (2 * n - 1);
        term = -term * x * x / denominador;
        cosx = cosx + term;
        n = n + 1;
    } while (presc <= fabs(cosval - cosx));
    printf("Suma de la serie coseno = %f\n", cosx);
    printf("Usando la libreria estandar cos(%d) = %f\n", x1, cos(x));
}
\end{minted}


\end{document}